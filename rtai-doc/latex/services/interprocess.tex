\chapter{Interprocess Communication}

\section{FIFOs}


\subsection{Overview}

In a well designed RTAI application you usually have a clear separation 
between time critical and non-critical parts of the code. This means
that you have to transport data from realtime land to userland and vice
versa. 

RTAI offers several interprocess communication mechanisms for this task.
One of the easiest method is the use of FIFOs. After a FIFO was created
you can simply "put" some data into it from one environment (e.\,g. from
userland) and read it later --~in the same order~-- from the other
envirionment (e.\,g. from kernel space). 


\subsection{FIFO creation}

To create a FIFO you can use this function: 

%\include{`$SOMEPATH/myscript -f rtai-core/rpc/fifos/fifos.c --function rtf_create --signature`}

%\inlinedoc{signature}{rtai-core/rpc/fifos/fifos.c}{rtf_create}

% Resultat: 

\begin{code}
	rtf_create (unsigned int minor, int size)
\end{code}

% ----------

%\include{`$SOMEPATH/myscript -f rtai-core/rpc/fifos/fifos.c --function rtf_create --complete`}

%\inlinedoc{manpage}{rtai-core/rpc/fifos/fifos.c}{rtf_create}

\textbf{NAME}

	\textbf{rtf\_create (unsigned int minor, int size)}

	minor: \& minor device number of the fifo
	size: \& buffer size for fifo data; this can be resized later
	with rtf\_resize()

\textbf{RETURN VALUE}

	0 on success
	-ENODEV illegal fifo minor number 
	-ENOMEM out of memory

% ----------

\subsection{Handlers}

\subsection{Signal Interface}

\subsection{Select/Poll and Blocking}

\subsection{Named FIFOs}

\subsection{API Reference}



\subsection{Implementation}

% ----------------------------------------------------------------------------

\section{Semaphores}

% These are now in FIFOs!
\subsection{Overview}

\subsection{API Reference}

% ----------------------------------------------------------------------------

\section{Shared Memory}

\subsection{Overview}

\subsection{Implementation}

\subsection{Build and Insert Shared Memory}

\subsection{Using Shared Memory in User Space}

\subsection{Using Shared Memory in Kernel Space}

\subsection{Proc}

\subsection{API Reference}

% ----------------------------------------------------------------------------

\section{Mailboxes}

\subsection{Overview}

\subsection{Implementation}

\subsection{API Reference}

% ----------------------------------------------------------------------------
\section{Extended Messages}

% Documentation/README.EXTMSG

\subsection{Overview}

\subsection{API Reference}

% ----------------------------------------------------------------------------

\section{Remote Procedure Call}

\subsection{Overview}

\subsection{API Reference}


