\chapter{Posix - RTAI/fusion}

% ----------------------------------------------------------------------------

\section{The New POSIX Approach}

In older RTAI versions POSIX compatiblity has been achieved by writing 
several modules which implemented the standard POSIX APIs in RTAI 
modules. This approach has turned out to be problematic and difficult
to maintain, so a new scheme has been invented. 

Instead of rewriting a complete POSIX layer usable in a real-time
context, a way has been found to move the existing user-space Linux
support seamlessly into the RTAI realm. This basically allows to call
regular Linux syscalls synchronously from RTAI tasks running in
user-space, while keeping the scheduling priority of the caller
unaffected by the domain migration.

This approach offers a number of advantages:

\begin{itemize}

\item Conflicts (source-wise and execution-wise) between the RTAI-specific
and native POSIX interfaces (Linuxthreads/NPTL) cannot occur, since
there is no need for any RTAI-specific interface in the first place.

\item Since user-space thread interfaces use kernel syscalls to perform
thread synchronization and other management, interface upgrades should
have very little impact on the RTAI POSIX support. Conversely, RTAI
should benefit from POSIX interface upgrades at low or even no cost.

\item All user-space services are now automatically RTAI-enabled,
i.e. real-time compatible, provided they are thread-safe in the first
place. Obviously, this does not mean that Linux services miraculously
become time-bounded with this extension, but at least, they are now
callable from any RTAI task context. It is still up to the user to
evaluate the suitability of calling such code in a real-time context
though.

\end{itemize}

% ----------------------------------------------------------------------------

\subsection{How it works}

LXRT is able to migrate user-space tasks back and forth between the
Linux and RTAI domains, depending on the requested operating mode.

Hard real-time mode is obtained by stealing the current task from the
Linux's scheduler runqueue and readying it for the LXRT
scheduler. Since the latter also reinstates the MMU-context of any
incoming task, the net result is that we have memory protected tasks
solely scheduled by RTAI. After this transition, the migrated/stolen
task left in \texttt{TASK\_UNINTERRUPTIBLE} state disappears from Linux's radar,
just like any suspended task, assuming that some code somewhere will
care for resuming it when it's due. At this point, the RTAI scheduler
can pick this task and reinstate its MMU-context when it switches it
back in.

Migration from Linux to RTAI needs a helper task controlled by Linux
(namely, \texttt{kthread\_b}) which is resumed by RTAI, and whose job is to move
any given real-time task to RTAI's ready task queue, then trigger a
fast real-time rescheduling to make such task resume into the RTAI
domain immediately. The action of waking up this helper task after the
migrating task has been suspended (state == \texttt{TASK\_UNINTERRUPTIBLE})
conveniently ensures that the Linux runqueue correctly reflects the
new situation.

The following pseudo-code example shows how it works: 

\begin{code}
T1 runs in soft real-time mode...
        T1 calls make_hard_realtime()
                T1 Linux state is set TASK_UNINTERRUPTIBLE
                kthread_b() kernel thread is woken up
kthread_b() runs... (T1 is now out of Linux's runqueue)
        kthread_b() fast schedules back T1
T1 immediately resumes in hard real-time mode...
\end{code}

Migration from RTAI to Linux is somewhat simpler, only requiring a
RTAI service request to be scheduled so that \texttt{wake\_up\_process()} is
called on the migrating task as soon as the Linux kernel gets back in
control.

% ----------------------------------------------------------------------------

\subsection{Side-effect of a soft-mode transition in LXRT}

Soft-mode transition is caused by an explicit call to
\texttt{make\_soft\_realtime()}, or implicitely done by LXRT when a Linux syscall
is caught from a hard real-time task. LXRT also makes sure that such
task goes back to hard real-time as soon as the Linux syscall has
completed.

The first effect of the soft-mode transition is to lower the calling
task's priority down to the Linux placeholder task
(i.e. \texttt{rt\_linux\_task}), representing the idle -- non real-time fallback
context. This is an accepted side-effect that such lowering also
affects the real-time priority handling, de facto excluding the soften
task from the real-time realm.

% ----------------------------------------------------------------------------

\section{Issues and solutions}

The following issues are raised when it comes to integrate the Linux
General Purpose Operating System (GPOS) kernel within the execution 
domain of a RTOS like RTAI:

\begin{itemize}

\item Linux internal design promotes fairness and throughput, at the
expense of determinism. The net effect is that many code paths in the
vanilla 2.4 Linux kernel are not preemptible for serving interrupts
and/or rescheduling tasks.

\end{itemize}

By increasing the overall preemptibility of the Linux
kernel, Montavista's "kpreempt" extension partly solves this
issue, by limiting the non-preemptible sections to those
protected by a spinlock, and calling the rescheduling
procedure on exit of such sections if needed. Coupling Andrew
Morton's low latency patch further reduces the length of such
non-preemptible sections.

\begin{itemize}

\item Fine grained Linux kernel preemptibility does not help prioritizing
the interrupt load from a real-time point of view, which means that low-priority
bottom-halves can still preempt high-priority real-time Linux tasks
(SCHED\_FIFO) under this model, thus introducing unbounded latencies.

\end{itemize}

By implementing a technique called "interrupt shielding"
using ADEOS capabilities, we can control the interrupt flow
so that Linux IRQ handlers are not fired while real-time
RTAI tasks tread on Linux kernel code. This is obtained by
inserting an intermediate ADEOS domain between RTAI and Linux,
and stalling/unstalling this stage at the proper time. This
way, whatever Linux actions are regarding its own interrupt
state, ADEOS guarantees that the effective interrupt control
for the GPOS kernel is left in the hands of the shielding
domain above Linux in the pipeline.

\begin{verbatim}
i.e. IRQ -> [RTAI] -->-- [Shield] -->-- [Linux] (std mode)

     IRQ -> [RTAI] -->-- [Shield] --/   [Linux] (rt mode)
\end{verbatim}

\begin{itemize}

\item Interrupt shielding has one drawback: RTAI tasks which end up
suspending themselves on a Linux kernel resource waiting for a Linux
interrupt to occur (e.g. I/O or timing wait) will be woken up only
after the shield is deactivated, i.e. when no other RTAI task is
running into the Linux kernel domain. This fact is a source of
priority inversions, since low priority RTAI tasks could prevent
hi-priority ones to wake up until they relinquish the processor.
Additionally, Linux's periodic timer resolution is not suitable for
hi-frequency real-time processing.

\end{itemize}

Timing syscalls (\texttt{gettimer()}/\texttt{setitimer()} and \texttt{nanosleep()}) used by
the POSIX layer are transparently intercepted by RTAI/fusion,
and converted to their RTAI-based hard real-time
counterparts. This conversion has two positive effects: RTAI's
timer resolution is much higher, and timed RTAI tasks
are suspended into the RTAI domain, thus cannot be shielded
from timer interrupts by the intermediate domain.

However, RTAI tasks waiting on I/O events only handled by the
Linux kernel are still subject to priority inversions. For
now, the best way to prevent this undesirable side-effect is
to have those tasks handled by RTAI I/O drivers instead of
relying on Linux drivers, since wakeups in the RTAI domain
are performed immediately by the RTAI interrupt handler.

\begin{itemize}

\item RTAI currently provides determinism \emph{aside} of Linux by adding a
preemptive and concurrent kernel. Services requests from RTAI to Linux
then cause the requested services to run outside the real-time
execution domain. RTAI tasks can only send asynchronous requests
(i.e. srqs) to perform Linux services which will be processed when no
real-time tasks are runnable.  Moreover, the service inherits the
scheduling priority of the last Linux task which happened to have been
preempted by RTAI. This situation currently prevents the Linux kernel
code to be part of the real-time execution domain.

\end{itemize}

By tracking the priority of the RTAI task running the Linux
kernel code and applying it dynamically to the Linux
placeholder task controlled by the RTAI scheduler
(i.e. rt\_linux\_task), the position of the RTAI task
within the real-time priority hierarchy is kept unchanged,
even when it executes into the Linux domain.

Additionally, the priority of the Linux task coupled to the
real-time RTAI task is set to a value preserving the overall
RTAI priority scheme.  This way, we have two schedulers
(i.e. RTAI and Linux) cooperating for the execution of
real-time tasks inside a single priority scheme across both
execution domains.

Example: 
\begin{verbatim}
RTAI domain   -- RTAI (pseudo-)task "T0" scheduled by rtai_sched
        (pri = 10)

Linux domain  -- Linux task "foo" (mapped to T0) scheduled by Linux
        (pri = sched_getpriority_max(SCHED_FIFO) + 1 + pri(T0))
\end{verbatim}
\begin{itemize}

\item When migrating tasks from the RTAI execution domain to the Linux
domain, a time-critical transition must occur, during which RTAI
delegates the control of this task to the Linux scheduler. This phase
must be fast and perfectly synchronized, to guarantee a safe
reentrancy into the Linux kernel code. Additionally, since the only
safe transition point is the Linux rescheduling procedure, the delay
incurred to reach the next rescheduling point inside any preemptible
portion of the Linux kernel must be bounded and as short as possible.

\end{itemize}

Transitions from RTAI to Linux are obtained by sending a
virtual ADEOS interrupt from the RTAI domain to the Linux
domain. The interrupt handler into the Linux kernel wakes up
the process as requested.

If the Linux kernel code was last preempted by RTAI outside
of a critical section, a Linux rescheduling takes place
immediately on the return path of the virtual
interrupt. Otherwise, the preemptible kernel will ensure that
a rescheduling takes place as soon as the critical section is
exited. With the additional benefit of the low latency patch,
those sections are kept as short as possible, enforcing
lock-breaking preemption points as needed.

% ----------------------------------------------------------------------------

\section{Practical Assumptions, Limits and Guarantees}

The new scheme has some practical assumptions and limits: 

\begin{itemize}

\item A vast majority of RTAI-based applications in user-space would rely
on POSIX services mainly for thread synchronization and hi-resolution
timing. Thread synchronization services would only pay the additional
cost of RTAI/Linux migrations - needed to re-enter the Linux kernel
from a RTAI task for performing the system call - when the access to
some resource is effectively contended by more than a single POSIX
thread. Provided this cost is at most a handful of microseconds, it is
assumed that the advantages of RTAI-enabling the Linux system calls
instead of rewriting specific APIs outweight this single drawback,
especially since most of the Linux syscalls issued for synchronization
by the native POSIX layer tend to put the caller to sleep. In any
case, very time-critical services could still be substituted with a
mere RTAI add-on if needed.

\item "In the POSIX interface we trust". IOW, it is assumed that the
standard POSIX interface used to program the real-time applications is
efficient as far as thread concurrency is concerned. As a consequence
of this, and aside of the kernel time consumed in executing syscalls
for its threads, rewriting another dedicated RTAI-specific interface
should not be significantly more efficient performance-wise.

\item "In the Linux kernel we trust". IOW, the syscalls internally used by
the standard POSIX interface for common operations are assumed to be
efficiently written and would not charge unacceptable CPU cost to any
application which currently targets user-space real-time support. The
preemptible and low latency kernel patches are expected to provide for
limited and bounded \_scheduling\_ latencies inside the Linux kernel.

\item Running a RTAI task into the Linux domain with an effective
real-time priority must not impact on the preemptability inside the
RTAI domain. The following assertions remain true at any point in
time:

        \begin{itemize}
        
        \item Adeos's pipelined interrupt model ensures that any unmasked
        interrupt is immediately dispatched to the RTAI domain if
        unstalled, regardless of the currently running domain (RTAI,
        shielding domain or Linux).

        \item Interrupt shielding ensures that Linux targeted interrupts
        are not propagated to the Linux handlers when a real-time
        task executes into the Linux kernel.

        \end{itemize}

\item The Linux kernel provides for a \texttt{SCHED\_FIFO} scheduling policy
enforcing static real-time priorities among the tasks it
controls. This is especially important as Linux's scheduling decisions
must not break the RTAI priority hierarchy when real-time tasks are
operating into the Linux domain.

\end{itemize}

RTAI/fusion will be adapted for running on the Linux 2.6 kernel in
the future, but as of now, please put on your 2.4 glasses when reading
this document.

% ----------------------------------------------------------------------------

\section{Implementation}

\subsection{Current Implementation}

RTAI/fusion is built upon the implementation of a mutable priority
scheme for the Linux placeholder task. As a remainder, this
pseudo-task currently represents all the non real-time Linux tasks,
and acts as a common fallback schedulable object to pick when no
real-time RTAI tasks are ready to run.

The test code is built upon the Xenomai nucleus. Xenomai provides the
scheduling of kernel-based and user-space threads, using a technology
derived from LXRT, and incorporating the implementation details listed
below (see 7). This nucleus includes a real-time, FIFO-based, static
priority scheduler.

Like LXRT, the Xenomai scheduler defines a Linux placeholder called
the "root" thread which gets scheduled as a fallback option when no
other Xenomai threads are runnable into the RTAI domain. This causes
the RTAI domain to suspend itself Adeos-wise, yielding control to the
next domain down the pipeline.

Each Xenomai thread capable of executing in user-space has one
kernel-based TCB called a "shadow", along with its standard
Linux-controlled task structure. Real-time scheduling of shadow
threads is done using those TCBs, but the register state is
exclusively saved into the Linux task structures by both Linux and
Xenomai schedulers during context switching. In Xenomai's terminology,
a shadow thread is the real-time control block of a Linux task also
schedulable by Xenomai. Both schedulers control the task in a mutually
exclusive manner.

The priority of the root thread is mutable, and can be changed
dynamically to reflect the position of the Linux kernel activity
within the real-time priority hierarchy.

A Linux-controlled gatekeeper task is in charge of resuming a Xenomai
thread migrating from Linux to the RTAI domain. This thread is the
equivalent of LXRT's kthread\_b() helper task.

Mutable priority works like this:

In a system defining T1(prio=1,domain=RTAI), T2(prio=2,domain=RTAI),
and ROOT(prio=0,domain=Linux), we would have the following transition:

\begin{code}
T2 runs in domain RTAI...
   T2 migrates to the Linux domain (for executing a syscall)
       ROOT inherits priority(T2)
                T2 executes the syscall on behalf of ROOT
                (at this point, interrupts are no more propagated to
                the Linux kernel)
   T2 migrates back to the RTAI domain (after the syscall completes)
        T2 resumes into the RTAI domain
        ROOT priority is lowered back (to 0 if no other real-time
        threads are runnable into the Linux domain, or inherited from the
        next real-time thread to execute in this domain). As soon as the
        root thread recovers its base priority (0), the interrupt flow is
        restarted toward the Linux kernel.
T2 suspends...
T1 resumes...
\end{code}

As shown, T2 migration would not cause the preemption of the root
thread by T1 despite its lower initial priority, since ROOT's priority
would not allow it after it has been dynamically raised.

In practice, T2 would not be forced to exit the Linux domain right
after the syscall completes, but only when it issues a RTAI
syscall. The migration back to RTAI has been described for
illustration purposes.

Obviously, ROOT never migrates to RTAI, and is always runnable.

% ----------------------------------------------------------------------------

\subsection{Implementation details with regard to priority mutation}

When informed by Adeos that a new Linux task is about to be switched
in (ADEOS\_SCHEDULE\_HEAD event), we check in the event handler whether
a shadow thread is paired with the incoming task.  In such a case, the
root thread priority is raised to the priority of the shadow,
otherwise, this priority is set to the "idle" priority level
(i.e. non-realtime priority).

In the RTAI code controlling the migration of the current thread
from RTAI to the Linux domain, the priority of the root thread is
inherited from the migrating thread.

In the RTAI code controlling the migration from the current Xenomai
thread from Linux to the RTAI domain, the priority of the gatekeeper
thread is inherited from the migrating thread.

\section{Hardware/Software pre-requisites}

The experimental RTAI/fusion subsystem currently runs on the following
configuration:

\begin{itemize}

\item Linux 2.4.21/ia32
\item ADEOS combo patch (Adeos + Kpreempt + Lolat)
\item Xenomai nucleus from the RTAI vesuvio branch.

\end{itemize}

A simple demo is available for download here:

\centerline{http://savannah.gnu.org/download/xenomai/fusion/}

% ----------------------------------------------------------------------------

\section{Programming with POSIX API}
\subsection{Threads}
\subsection{...}
\subsection{API Reference}

\section{Interprocess Communication}
\subsection{Overview}
\subsection{Mutexes}
\subsection{Conditional Variables}
\subsection{Queues}
\subsection{API Reference}
